\documentclass[11pt, a4paper]{article}
\usepackage{authblk}

\usepackage{amsmath}
\usepackage{amssymb}

\usepackage{natbib}
\bibliographystyle{plainnat}

\def\MM#1{\boldsymbol{#1}}

\begin{document}

\author[1,*]{Jemma Shipton}
\author[2]{Thomas M. Bendall}
\author[3]{So many others}

\affil[1]{Department of Mathematics and Statistics, University of Exeter}
\affil[2]{Met Office}
\affil[*]{Correspondence to: \texttt{j.shipton@exeter.ac.uk}}

\title{Gusto: a toolkit for compatible finite element dynamical cores}
\date{}

\maketitle

\section{Background}



\subsection{Motivation}
\begin{itemize}
\item route to discretisation of eqns with correct properties on non-uniform grids
\item these properties are considered essential for dynamical cores
\item non-uniform grids are considered important for parallel scaling
\item many choices and research to be done therefore flexible implementation is useful
\end{itemize}

Compatible finite elements provide a route to stable, consistent
discretisations of the partial differential equations governing
geophysical fluids which also have the correct wave propagation
properties on non-uniform grids and avoid spurious computational
modes. These properties, along with conservation of mass and
potentially other quantities such as energy and enstrophy, are
considered essential for the accuracy of the dynamical core of weather
and climate models. While the wave propagation and required
conservation properties guide the choice of finite element function
spaces for the discretisation, there are many options for both the
spatial and temporal discretisation and it is not clear which will be
best for any particular problem. Gusto provides a flexible,
easy-to-use toolkit of options for rapid prototyping of different
spatial and temporal discretisations for geophysical fluid dynamics
equations relevant to numerical weather and climate prediction.

In this article we will describe the current capabilities of Gusto and
the design features that enable our flexible implementation whereby
different spatial and temporal discretisations can be applied to a
range of different equations sets relevant for geophysical fluid
modelling. In section \ref{sec: governing} we will outline three
different equation sets, along with their finite element
discretisations. Then in section \ref{sec: design} we will present the
code structure of Gusto followed by a detailed example (section
\ref{sec: FML}) of the Form Manipulation Language that facilitates
automatic manipulation of the finite element forms within our
timestepping code. Section \ref{sec: results} contains a selection of
numerical results that showcases the range of discretisations
available within Gusto. We summarise our current status and future
directions in section \ref{sec: summary}.

\section{Governing Equations}
\label{sec: governing}

The aim of this section is to introduce three governing equation sets
commonly used in the development of weather and climate models, along
with enough detail about the finite element discretiations provided by
Gusto to demonstrate the similarities between them. These similarities
guide the design of the code structure described in the next section
and motivate the use of Form Manipulation Labelling (FML) that
underlies the flexibility of Gusto, a detailed example of which will
be given in section \ref{sec: FML}.

\subsection{Shallow Water Equations}
The shallow water equations are commonly used for testing numerical
algorithms and for exploring geophysical fluid dynamics concepts under
simplified conditions. They are useful in both contexts since they are
the simplest set of equations that model motion on both the slow,
geostrophic timescale and the much faster timescale of the
inertia-gravity waves. This timescale separation poses challenges to
the numerical discretisation and also provides a rich enough range of
dynamics to explain many of the phenomena observed in the atmosphere
and oceans \citep{zeitlin2018geophysical}.

The shallow water equations describe the flow of a shallow layer of
fluid subject to gravitational and, optionally, rotational forces. The
prognostic variables are the two horizontal velocity components
$\MM{u} = (u, v)$ and the fluid depth $D = H + h - b$ where $H$ is the
undisturbed depth of the fluid layer, $h$ is the perturbation to the
fluid depth and $b$ is the bottom topography, if present. The
equations are:

\begin{align}
  \frac{\partial\MM{u}}{\partial t} + \MM{u}\cdot\nabla\MM{u} + f\MM{k}\times\MM{u} + g\nabla (D+b) &= 0, \\
  \frac{\partial D}{\partial t} + \nabla\cdot(\MM{u}D) &= 0,
\end{align}
where $f$ is the Coriolis parameter, $\MM{k}$ is the vertical vector
normal to the domain and $g$ is the acceleration due to gravity. The
nonlinear velocity advection term can be replaced by the sum of a
vorticity-based term and the gradient of the kinetic energy. This is
known as the vector invariant form and it enables the construction of
schemes that conservation energy and/or enstrophy
\citep{mcrae2014energy, bauer2018energy, wimmer2020energy,
  wimmer2021energy}. However, this form can exhibit numerical
instabilities \citep{bell2017numerical} and it is not yet clear which
form is preferable so in Gusto we offer both forms of the equations.

The finite element discretisation of these equations is formed by
choosing appropriate function spaces for the prognostic variables,
multiplying by test functions from these function spaces and
integrating over the domain. In addition to this, we integrate by
parts when the inter-element continuity of the basis functions is not
sufficient to define the result of applying the differential operator
directly. In line with the theory outlined in section \ref{}, we
choose $\MM{u} \in \mathbb{V}_1$ and $D \in \mathbb{V}_2$ with
$\mathbb{V}_1 \subset H(\text{div})$ and $\mathbb{V}_2 \subset L^2$. The
corresponding test functions are denoted by $\MM{w}$ and $\phi$. This gives

\begin{align}
  \int_\Omega\MM{w}\cdot\frac{\partial\MM{u}}{\partial t}dV + \int_\Omega\MM{w}\cdot(\MM{u}\cdot\nabla)\MM{u} dV + \int_\Omega\MM{w}\cdot f\MM{k}\times\MM{u}dV - \int_\Omega \nabla\cdot\MM{w} g(D+b) dV &= 0, \forall \MM{w}\in\mathbb{V}_1 \\
  \int_\Omega\phi\frac{\partial D}{\partial t}dV + \int_\Omega\phi\nabla\cdot(\MM{u}D) dV &= 0, \forall \phi\in\mathbb{V}_2
\end{align}

\subsection{Compressible Euler Equations}
The dynamical core at the heart of any weather or climate model solves
the rotating compressible Euler equations that model atmospheric flow,
often with additional approximations such as shallow vs deep
atmosphere or hydrostatic vs non-hydrostatic flow. These equations can
be written in terms of

\subsection{Compressible Boussinesq Equations}

\section{Software Design}
\label{sec: design}

\begin{itemize}
\item include schematic
\item generic example - no FML
\item lead in to next section... facilitated by FML
\end{itemize}

\section{Form Manipulation Labelling}
\label{sec: FML}

\section{Results}
\label{sec: results}

The aim of this section is to present a selection of results that
demonstrate the tools available in Gusto, such as the choice of grids,
different discretisation options, timestepping schemes and
physics. The numerical properties of these configurations
(convergence, stability, accuracy and efficiency) have either been
demonstrated elsewhere, or will be investigated more thoroughly in
future publications.

\begin{itemize}
\item grids
\item FE order
\item spatial discretisation options
\item timestepping schemes
\item physics
\end{itemize}

\subsection{Shallow water simulations}
The standard suite of shallow water test cases originally proposed in
\citet{williamson1992standard}



\section{Summary}
\label{sec: summary}

\section{Code availability}

\bibliography{references}

\end{document}
