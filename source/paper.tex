\documentclass[11pt, a4paper]{article}
\usepackage{authblk}

\usepackage{amsmath}
\usepackage{amssymb}

\usepackage{natbib}
\bibliographystyle{plainnat}

\def\MM#1{\boldsymbol{#1}}

\begin{document}

\author[1,*]{Jemma Shipton}
\author[2]{Thomas M. Bendall}
\author[3]{So many others}

\affil[1]{Department of Mathematics and Statistics, University of Exeter}
\affil[2]{Met Office}
\affil[*]{Correspondence to: \texttt{j.shipton@exeter.ac.uk}}

\title{Gusto: a toolkit for compatible finite element dynamical cores}
\date{}

\maketitle

\section{Background}

\subsection{Overview}

\subsection{Motivation}
Compatible finite elements provide a route to stable discretisations
of the partial differential equations governing geophysical fluids
which also have the correct wave propagation properties on non-uniform
grids. While the wave propagation and required conservation properties
guide the choice of finite element function spaces for the
discretisation, there are many options for both the spatial and
temporal discretisation and it is not clear which will be best for any
particular problem. Gusto provides a flexible, easy-to-use toolkit of
options for rapid prototyping of different spatial and temporal
discretisations for geophysical fluid dynamics equations relevant to
numerical weather and climate prediction.

\section{Governing Equations}

\subsection{Shallow Water Equations}
The shallow water equations are commonly used for testing numerical
algorithms and for exploring geophysical fluid dynamics concepts under
simplified conditions. They are useful in both contexts since they are
the simplest set of equations that model motion on both the slow,
geostrophic timescale and the much faster timescale of the
inertia-gravity waves. This timescale separation poses challenges to
the numerical discretisation and also provides a rich enough range of
dynamics to explain many of the phenomena observed in the atmosphere
and oceans \citep{zeitlin2018geophysical}.

The shallow water equations describe the flow of a shallow layer of
fluid subject to gravitational and, optionally, rotational forces. The
prognostic variables are the two horizontal velocity components
$\MM{u} = (u, v)$ and the fluid depth $D = H + h - b$ where $H$ is the
undisturbed depth of the fluid layer, $h$ is the perturbation to the
fluid depth and $b$ is the bottom topography, if present. The
equations are:

\begin{align*}
  \frac{\partial\MM{u}}{\partial t} + \MM{u}\cdot\nabla\MM{u} + f\MM{k}\times\MM{u} + g\nabla (D+b) &= 0, \\
  \frac{\partial D}{\partial t} + \nabla\cdot(\MM{u}D) &= 0,
\end{align*}
where $f$ is the Coriolis parameter, $\MM{k}$ the vertical vector normal
to the domain and $g$ the acceleration due to gravity. The nonlinear velocity
advection term can be replaced by the sum of a vorticity-based term
and the gradient of the kinetic energy. This is known as the vector
invariant form and it enables the construction of vorticity based
advection schemes that lead to conservation of energy and/or enstrophy
(**refs). However, this form can exhibit numerical instabilities and
it is not yet clear which is preferable so in Gusto we offer both
forms of the equations.

The finite element discretisation of these equations is formed by
choosing appropriate function spaces for the prognostic variables,
multiplying by test functions from these function spaces and
integrating over the domain. In addition to this, we integrate by
parts when the inter-element continuity of the basis functions is not
sufficient to define the result of applying the differential operator
directly. In line with the theory outlined in section \ref{}, we
choose $\MM{u} \in \mathbb{V}_1$ and $D \in \mathbb{V}_2$ with
$\mathbb{V}_1 \subset H(\text{div})$ and $\mathbb{V}_2 \subset L^2$. The
corresponding test functions are denoted by $\MM{w}$ and $\phi$. This gives

\begin{align*}
  \int_\Omega\MM{w}\cdot\frac{\partial\MM{u}}{\partial t}dV + \MM{u}\cdot\nabla\MM{u} + f\MM{k}\times\MM{u} - \int_\Omega \nabla\cdot\MM{w} g(D+b) dV &= 0, \forall \MM{w}\in\mathbb{V}_1 \\
  \int_\Omega\phi\frac{\partial D}{\partial t}dV + \int_\Omega\phi\nabla\cdot(\MM{u}D) dV &= 0, \forall \phi\in\mathbb{V}_2
\end{align*}

\subsection{Compressible Euler Equations}

\subsection{Incompressible Boussinesq Equations}

\section{Gusto Design}

\begin{itemize}
\item include schematic
\item generic example - no FML
\item lead in to next section... facilitated by FML
\end{itemize}

\section{Form Manipulation Labelling}

\section{Results}
The aim of this section is to present a selection of results that
demonstrate the tools available in Gusto, such as the choice of grids,
different discretisation options, timestepping schemes and
physics. The numerical properties of these configurations
(convergence, stability, accuracy and efficiency) have either been
demonstrated elsewhere, or will be investigated more thoroughly in
future publications.

\begin{itemize}
\item grids
\item FE order
\item spatial discretisation options
\item timestepping schemes
\item physics
\end{itemize}

\subsection{Shallow water simulations}
The standard suite of shallow water test cases originally proposed in
\citet{williamson1992standard}



\section{Summary}

\section{Code availability}

\bibliography{references}

\end{document}
